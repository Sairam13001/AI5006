\documentclass[journal,12pt,twocolumn]{IEEEtran}
%
\usepackage{setspace}
\usepackage{gensymb}
%\doublespacing
\singlespacing

\usepackage{graphicx}
\usepackage[cmex10]{amsmath}
\usepackage{amsmath,amsthm}
\usepackage{mathrsfs}
\usepackage{txfonts}
\usepackage{stfloats}
\usepackage{bm}
\usepackage{cite}
\usepackage{cases}
\usepackage{subfig}

\usepackage{longtable}
\usepackage{multirow}
\usepackage{commath}
\usepackage{enumitem}
\usepackage{mathtools}
\usepackage{steinmetz}
\usepackage{tikz}
\usepackage{circuitikz}
\usepackage{verbatim}
\usepackage{tfrupee}
\usepackage[breaklinks=true]{hyperref}

\usepackage{tkz-euclide}

\usetikzlibrary{calc,math}
\usepackage{listings}
    \usepackage{color}                                            
    \usepackage{array}                                            
    \usepackage{longtable}                                        
    \usepackage{calc}                                             
    \usepackage{multirow}                                         
    \usepackage{hhline}                                           
    \usepackage{ifthen}                                           
    \usepackage{lscape}     
\usepackage{multicol}
\usepackage{chngcntr}

\DeclareMathOperator*{\Res}{Res}

\renewcommand\thesection{\arabic{section}}
\renewcommand\thesubsection{\thesection.\arabic{subsection}}
\renewcommand\thesubsubsection{\thesubsection.\arabic{subsubsection}}

\renewcommand\thesectiondis{\arabic{section}}
\renewcommand\thesubsectiondis{\thesectiondis.\arabic{subsection}}
\renewcommand\thesubsubsectiondis{\thesubsectiondis.\arabic{subsubsection}}

\hyphenation{op-tical net-works semi-conduc-tor}
\def\inputGnumericTable{}                                 

\lstset{
%language=C,
frame=single, 
breaklines=true,
columns=fullflexible
}
\lstset{
%language=TeX,
frame=single, 
breaklines=true
}

\begin{document}


\newtheorem{theorem}{Theorem}[section]
\newtheorem{problem}{Problem}
\newtheorem{proposition}{Proposition}[section]
\newtheorem{lemma}{Lemma}[section]
\newtheorem{corollary}[theorem]{Corollary}
\newtheorem{example}{Example}[section]
\newtheorem{definition}[problem]{Definition}

\newcommand{\BEQA}{\begin{eqnarray}}
\newcommand{\EEQA}{\end{eqnarray}}
\newcommand{\define}{\stackrel{\triangle}{=}}
\bibliographystyle{IEEEtran}
\providecommand{\mbf}{\mathbf}
\providecommand{\pr}[1]{\ensuremath{\Pr\left(#1\right)}}
\providecommand{\qfunc}[1]{\ensuremath{Q\left(#1\right)}}
\providecommand{\sbrak}[1]{\ensuremath{{}\left[#1\right]}}
\providecommand{\lsbrak}[1]{\ensuremath{{}\left[#1\right.}}
\providecommand{\rsbrak}[1]{\ensuremath{{}\left.#1\right]}}
\providecommand{\brak}[1]{\ensuremath{\left(#1\right)}}
\providecommand{\lbrak}[1]{\ensuremath{\left(#1\right.}}
\providecommand{\rbrak}[1]{\ensuremath{\left.#1\right)}}
\providecommand{\cbrak}[1]{\ensuremath{\left\{#1\right\}}}
\providecommand{\lcbrak}[1]{\ensuremath{\left\{#1\right.}}
\providecommand{\rcbrak}[1]{\ensuremath{\left.#1\right\}}}
\theoremstyle{remark}
\newtheorem{rem}{Remark}
\newcommand{\sgn}{\mathop{\mathrm{sgn}}}
\providecommand{\abs}[1]{\(\left\vert#1\right\vert\)}
\providecommand{\res}[1]{\Res\displaylimits_{#1}} 
\providecommand{\norm}[1]{\(\left\lVert#1\right\rVert\)}
%\providecommand{\norm}[1]{\lVert#1\rVert}
\providecommand{\mtx}[1]{\mathbf{#1}}
\providecommand{\mean}[1]{E\(\left[ #1 \right]\)}
\providecommand{\fourier}{\overset{\mathcal{F}}{ \rightleftharpoons}}
%\providecommand{\hilbert}{\overset{\mathcal{H}}{ \rightleftharpoons}}
\providecommand{\system}{\overset{\mathcal{H}}{ \longleftrightarrow}}
	%\newcommand{\solution}[2]{\textbf{Solution:}{#1}}
\newcommand{\solution}{\noindent \textbf{Solution: }}
\newcommand{\cosec}{\,\text{cosec}\,}
\providecommand{\dec}[2]{\ensuremath{\overset{#1}{\underset{#2}{\gtrless}}}}
\newcommand{\myvec}[1]{\ensuremath{\begin{psmallmatrix}#1\end{psmallmatrix}}}
\newcommand{\mydet}[1]{\ensuremath{\begin{vmatrix}#1\end{vmatrix}}}
%\numberwithin{equation}{section}
\numberwithin{equation}{subsection}
%\numberwithin{problem}{section}
%\numberwithin{definition}{section}
\makeatletter
\@addtoreset{figure}{problem}
\makeatother
\let\StandardTheFigure\thefigure
\let\vec\mathbf
%\renewcommand{\thefigure}{\theproblem.\arabic{figure}}
\renewcommand{\thefigure}{\theproblem}
%\setlist[enumerate,1]{before=\renewcommand\theequation{\theenumi.\arabic{equation}}
%\counterwithin{equation}{enumi}
%\renewcommand{\theequation}{\arabic{subsection}.\arabic{equation}}
\def\putbox#1#2#3{\makebox[0in][l]{\makebox[#1][l]{}\raisebox{\baselineskip}[0in][0in]{\raisebox{#2}[0in][0in]{#3}}}}
     \def\rightbox#1{\makebox[0in][r]{#1}}
     \def\centbox#1{\makebox[0in]{#1}}
     \def\topbox#1{\raisebox{-\baselineskip}[0in][0in]{#1}}
     \def\midbox#1{\raisebox{-0.5\baselineskip}[0in][0in]{#1}}
\vspace{3cm}
\title{Assignment 3}
\author{Sairam V C Rebbapragada}
\maketitle
\newpage
%\tableofcontents
\bigskip
\renewcommand{\thefigure}{\theenumi}
\renewcommand{\thetable}{\theenumi}
\begin{abstract}
This document explains the concepts of Matrix multiplication, Matrix Addition and Matrix Inverse by solving a problem.
\end{abstract}
Download the python code from 
%
\begin{lstlisting}
https://github.com/Sairam13001/AI5006/blob/master/Assignment_2/assignment_2.py
\end{lstlisting}
%
and latex-tikz codes from 
%
\begin{lstlisting}
https://github.com/Sairam13001/AI5006/blob/master/Assignment_2/assignment_2.tex
\end{lstlisting}
%
\section{Problem}
\begin{align*}
   \vec{A}  = \myvec{ 3 & 1\\ -1 & 2}     
\end{align*}
Show that $\vec{A}^2$ - 5$\vec{A}$ + 7$\vec{I}$ = 0. Hence find $\Vec{A}^{-1}$.


\section{Explanation}

Square of a matrix is the product of matrix with itself :

\begin{align}
\Vec{A}^2 = \vec{A}.\Vec{A}
\end{align}

Product of a scalar with a matrix is the product of that scalar with every element of the matrix : 
\begin{align}
 k\myvec{ a & b\\c & d}  = \myvec{ ka & kb\\kc & kd}  
\end{align}

Inverse of a matrix $\vec{A}$ is defined as  :
  
\begin{align}
    \vec{A} . \vec{A}^{-1} = \vec{I}
\end{align}

\section{Solution}


Square of the given matrix A is : 
\begin{align}
   \vec{A}^2  = \myvec{ 3 & 1\\ -1 & 2}  \myvec{ 3 & 1\\ -1 & 2} =   \myvec{ 8 & 5\\ -5 & 3} 
\end{align}

$\vec{A}^2$ - 5$\vec{A}$ + 7$\vec{I}$ :
\begin{align}
  \myvec{ 8 & 5\\ -5 & 3} -  5\myvec{ 3 & 1\\ -1 & 2} + 7\myvec{ 1 & 0\\ 0 & 1}
\end{align}

\begin{align}
 =   \myvec{ 8 & 5\\ -5 & 3} -  \myvec{ 15 & 5\\ -5 & 10} + \myvec{ 7 & 0\\ 0 & 7}
\end{align}

\begin{align}
 =   \myvec{ 8 & 5\\ -5 & 3} - \myvec{ 8 & 5\\ -5 & 3} = \myvec{ 0 & 0\\ 0 & 0} 
\end{align}

Thus, It is proved that

\begin{align}
\vec{A}^2 - 5\vec{A} + 7\vec{I} = 0.  
\end{align}

Multiplying equation \brak{3.0.5} with $\Vec{A}^{-1}$ on both sides, We get :
\begin{align}
\Vec{A}^{-1}\brak{\vec{A}^2 - 5\vec{A} + 7\vec{I}} &= 0.\Vec{A}^{-1} \\
\implies \vec{A}^2.\Vec{A}^{-1} - 5\vec{A}.\Vec{A}^{-1} + 7\vec{I}.\Vec{A}^{-1} &= 0 \\
\implies \vec{A}.\vec{A}.\Vec{A}^{-1} - 5\vec{I} + 7\Vec{A}^{-1} &= 0 \\
\implies \vec{A}.\vec{I} - 5\vec{I} + 7\Vec{A}^{-1} &= 0\\
\implies \Vec{A}^{-1} &= \frac{1}{7}\brak{5\vec{I} - \vec{A}}
\end{align}

Solving for $\Vec{A}^{-1}$, we get :
\begin{align}
    \Vec{A}^{-1} &= \frac{1}{7}\brak{5\myvec{ 1 & 0\\ 0 & 1} - \myvec{ 3 & 1\\ -1 & 2}} \\
    \implies \Vec{A}^{-1} &= \frac{1}{7}\brak{\myvec{ 2 & -1\\ 1 & 3}}
\end{align}

\end{document}