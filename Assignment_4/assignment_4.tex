\documentclass[journal,12pt,twocolumn]{IEEEtran}
%
\usepackage{setspace}
\usepackage{gensymb}
%\doublespacing
\singlespacing

%\usepackage{graphicx}
%\usepackage{amssymb}
%\usepackage{relsize}
\usepackage[cmex10]{amsmath}
%\usepackage{amsthm}
%\interdisplaylinepenalty=2500
%\savesymbol{iint}
%\usepackage{txfonts}
%\restoresymbol{TXF}{iint}
%\usepackage{wasysym}
\usepackage{amsthm}
%\usepackage{iithtlc}
\usepackage{mathrsfs}
\usepackage{txfonts}
\usepackage{stfloats}
\usepackage{bm}
\usepackage{cite}
\usepackage{cases}
\usepackage{subfig}
%\usepackage{xtab}
\usepackage{longtable}
\usepackage{multirow}
%\usepackage{algorithm}
%\usepackage{algpseudocode}
\usepackage[utf8]{inputenc}
\usepackage{tikz}
\usetikzlibrary{positioning}
\usepackage{enumitem}
\usepackage{mathtools}
\usepackage{steinmetz}
\usepackage{tikz}
\usepackage{circuitikz}
\usepackage{verbatim}
\usepackage{tfrupee}
\usepackage[breaklinks=true]{hyperref}
%\usepackage{stmaryrd}
\usepackage{tkz-euclide} % loads  TikZ and tkz-base
%\usetkzobj{all}
\usetikzlibrary{calc,math}
\usepackage{listings}
    \usepackage{color}                                            %%
    \usepackage{array}                                            %%
    \usepackage{longtable}                                        %%
    \usepackage{calc}                                             %%
    \usepackage{multirow}                                         %%
    \usepackage{hhline}                                           %%
    \usepackage{ifthen}                                           %%
  %optionally (for landscape tables embedded in another document): %%
    \usepackage{lscape}     
\usepackage{multicol}
\usepackage{chngcntr}
%\usepackage{enumerate}

%\usepackage{wasysym}
%\newcounter{MYtempeqncnt}
\DeclareMathOperator*{\Res}{Res}
%\renewcommand{\baselinestretch}{2}
\renewcommand\thesection{\arabic{section}}
\renewcommand\thesubsection{\thesection.\arabic{subsection}}
\renewcommand\thesubsubsection{\thesubsection.\arabic{subsubsection}}

\renewcommand\thesectiondis{\arabic{section}}
\renewcommand\thesubsectiondis{\thesectiondis.\arabic{subsection}}
\renewcommand\thesubsubsectiondis{\thesubsectiondis.\arabic{subsubsection}}

% correct bad hyphenation here
\hyphenation{op-tical net-works semi-conduc-tor}
\def\inputGnumericTable{}                                 %%

\lstset{
%language=C,
frame=single, 
breaklines=true,
columns=fullflexible
}
%\lstset{
%language=tex,
%frame=single, 
%breaklines=true
%}

\begin{document}
%


\newtheorem{theorem}{Theorem}[section]
\newtheorem{problem}{Problem}
\newtheorem{proposition}{Proposition}[section]
\newtheorem{lemma}{Lemma}[section]
\newtheorem{corollary}[theorem]{Corollary}
\newtheorem{example}{Example}[section]
\newtheorem{definition}[problem]{Definition}
%\newtheorem{thm}{Theorem}[section] 
%\newtheorem{defn}[thm]{Definition}
%\newtheorem{algorithm}{Algorithm}[section]
%\newtheorem{cor}{Corollary}
\newcommand{\BEQA}{\begin{eqnarray}}
\newcommand{\EEQA}{\end{eqnarray}}
\newcommand{\define}{\stackrel{\triangle}{=}}
\bibliographystyle{IEEEtran}
%\bibliographystyle{ieeetr}
\providecommand{\mbf}{\mathbf}
\providecommand{\pr}[1]{\ensuremath{\Pr\left(#1\right)}}
\providecommand{\qfunc}[1]{\ensuremath{Q\left(#1\right)}}
\providecommand{\sbrak}[1]{\ensuremath{{}\left[#1\right]}}
\providecommand{\lsbrak}[1]{\ensuremath{{}\left[#1\right.}}
\providecommand{\rsbrak}[1]{\ensuremath{{}\left.#1\right]}}
\providecommand{\brak}[1]{\ensuremath{\left(#1\right)}}
\providecommand{\lbrak}[1]{\ensuremath{\left(#1\right.}}
\providecommand{\rbrak}[1]{\ensuremath{\left.#1\right)}}
\providecommand{\cbrak}[1]{\ensuremath{\left\{#1\right\}}}
\providecommand{\lcbrak}[1]{\ensuremath{\left\{#1\right.}}
\providecommand{\rcbrak}[1]{\ensuremath{\left.#1\right\}}}
\theoremstyle{remark}
\newtheorem{rem}{Remark}
\newcommand{\sgn}{\mathop{\mathrm{sgn}}}
\providecommand{\abs}[1]{\ensuremath{\left\vert#1\right\vert}}
\providecommand{\res}[1]{\Res\displaylimits_{#1}} 
\providecommand{\norm}[1]{\ensuremath{\left\lVert#1\right\rVert}}
%\providecommand{\norm}[1]{\lVert#1\rVert}
\providecommand{\mtx}[1]{\mathbf{#1}}
\providecommand{\mean}[1]{\ensuremath{E\left[ #1 \right]}}
\providecommand{\fourier}{\overset{\mathcal{F}}{ \rightleftharpoons}}
%\providecommand{\hilbert}{\overset{\mathcal{H}}{ \rightleftharpoons}}
\providecommand{\system}{\overset{\mathcal{H}}{ \longleftrightarrow}}
	%\newcommand{\solution}[2]{\textbf{Solution:}{#1}}
\newcommand{\solution}{\noindent \textbf{Solution: }}
\newcommand{\cosec}{\,\text{cosec}\,}
\providecommand{\dec}[2]{\ensuremath{\overset{#1}{\underset{#2}{\gtrless}}}}
\newcommand{\myvec}[1]{\ensuremath{\begin{pmatrix}#1\end{pmatrix}}}
\newcommand{\mydet}[1]{\ensuremath{\begin{vmatrix}#1\end{vmatrix}}}
%\numberwithin{equation}{section}
\numberwithin{equation}{subsection}
%\numberwithin{problem}{section}
%\numberwithin{definition}{section}
\makeatletter
\@addtoreset{figure}{problem}
\makeatother
\let\StandardTheFigure\thefigure
\let\vec\mathbf
%\renewcommand{\thefigure}{\theproblem.\arabic{figure}}
\renewcommand{\thefigure}{\theproblem}
%\setlist[enumerate,1]{before=\renewcommand\theequation{\theenumi.\arabic{equation}}
%\counterwithin{equation}{enumi}
%\renewcommand{\theequation}{\arabic{subsection}.\arabic{equation}}
\def\putbox#1#2#3{\makebox[0in][l]{\makebox[#1][l]{}\raisebox{\baselineskip}[0in][0in]{\raisebox{#2}[0in][0in]{#3}}}}
     \def\rightbox#1{\makebox[0in][r]{#1}}
     \def\centbox#1{\makebox[0in]{#1}}
     \def\topbox#1{\raisebox{-\baselineskip}[0in][0in]{#1}}
     \def\midbox#1{\raisebox{-0.5\baselineskip}[0in][0in]{#1}}
\vspace{3cm}
\title{Assignment 4}
\author{Sairam V C Rebbapragada}
\maketitle
\newpage
%\tableofcontents
\bigskip
\renewcommand{\thefigure}{\theenumi}
\renewcommand{\thetable}{\theenumi}
\begin{abstract}
This document uses the concepts of Inner product and Cosines in proving a statement.
\end{abstract}
Download Python code from 
%
\begin{lstlisting}
https://github.com/Sairam13001/AI5006/blob/master/Assignment_4/assignment_4.py
\end{lstlisting}
%
Download latex-tikz codes from 
%
\begin{lstlisting}
https://github.com/Sairam13001/AI5006/blob/master/Assignment_4/assignment_4.tex
\end{lstlisting}
%

\section{Problem}
$D$ is a point on side $BC$ of a $\triangle ABC$ such that $\frac{BD}{CD} =  \frac{AB}{AC}$. Prove that $AD$ is the bisector of $\angle BAC$ 

\section{Explanation}

If cosines of two angles are equal and those angles are less than 180\degree , then we can say that the angles are also equal.


\section{Solution}

\begin{figure}[!ht]
\centering
\resizebox{\columnwidth}{!}{
\begin{tikzpicture}
\draw (4,4) node[anchor=south]{$A$}
   -- (0,0) node[anchor=north]{$B$}
   -- (4,0) node[anchor=north]{$D$}
   -- (8,0) node[anchor=north]{$C$}
  -- cycle 
     (4,4) -- (4,0) ;
\end{tikzpicture}

}
\caption{$\triangle{ABC}$ with $D$ as a point on side $BC$}
\label{fig:triangle}
\end{figure}


See Fig. \ref{fig:triangle}, It is given that :
\begin{align}\label{givencond}
  \frac{\norm{\vec{B-D}}}{\norm{\vec{C-D}}} =  \frac{\norm{\vec{A-B}}}{\norm{\vec{A-C}}}
\end{align}
Re-writing the above equation we get :
\begin{align}\label{giveneqnedited}
  \frac{\norm{\vec{A-C}}}{\norm{\vec{A-B}}} =  \frac{\norm{\vec{C-D}}}{\norm{\vec{B-D}}}
\end{align}
Taking inner product of the sides $AB$,$AD$ we get :
\begin{align}\label{innerprod1}
     \cos{BAD} = \frac{(\vec{A}-\vec{B})^T(\vec{A}-\vec{D})}{\norm{\vec{A}-\vec{B}}\norm{\vec{A}-\vec{D}}} 
\end{align}
Taking inner product of the sides $AC$,$AD$ we get :
\begin{align}\label{innerprod2}
     \cos{CAD} = \frac{(\vec{A}-\vec{C})^T(\vec{A}-\vec{D})}{\norm{\vec{A}-\vec{C}}\norm{\vec{A}-\vec{D}}} 
\end{align}
Let $A = \myvec{0\\0}$, then equation \brak{\ref{innerprod1}} becomes:
\begin{align}\label{innerprod1.1}
     \cos{BAD} = \frac{(\vec{B})^T(\vec{D})}{\norm{\vec{B}}\norm{\vec{D}}} 
\end{align}
Similarly, equation \brak{\ref{innerprod2}} becomes :
\begin{align}\label{innerprod2.2}
     \cos{CAD} = \frac{(\vec{C})^T(\vec{D})}{\norm{\vec{C}}\norm{\vec{D}}} 
\end{align}
Dividing equation \brak{\ref{innerprod1.1}} by \brak{\ref{innerprod2.2}}, we get : 
\begin{align}\label{innerproddiv}
     \frac{\cos{BAD}}{\cos{CAD}} = \frac{(\vec{B})^T(\vec{D})\norm{\vec{C}}\norm{\vec{D}}}{(\vec{C})^T(\vec{D})\norm{\vec{B}}\norm{\vec{D}}}
\end{align}

In a triangle that satisfies the equation \brak{\ref{givencond}}, It can be seen that the equation \brak{\ref{innerproddiv}} always equates to 1, which implies that :
\begin{align}
    \cos{BAD} = \cos{CAD} \implies \angle BAD = \angle CAD.
\end{align}

Hence it is proved that $AD$ is the angle bisector of $\angle BAC$
\end{document}